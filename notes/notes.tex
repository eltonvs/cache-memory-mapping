\documentclass[12pt,openright,oneside,a4paper,english,brazil]{abntex2}

\usepackage[brazil]{babel}
\usepackage[utf8]{inputenc}
\usepackage{indentfirst}
\usepackage{listings}
\usepackage{color}
\usepackage{xcolor}

\usepackage{hyperref}
\hypersetup{
colorlinks = true,
citecolor  = gray,
linkcolor  = {blue!60!green}
}

% Creates \code tag to print inline code
\definecolor{codegray}{gray}{0.9}
\newcommand{\code}[1]{\colorbox{codegray}{\texttt{#1}}}

% Creates the raw environment
\newenvironment{raw}{\ttfamily}

% Document Info
\author{Elton de Souza Vieira\thanks{eltonviana@ufrn.edu.br || eltonviana.com}}
\title{Simulador de Associatividade Adaptativa}
\date{2016}
\local{Natal - RN}
\tipotrabalho{relatorio}

% PDF Metadata
\hypersetup{
    pdfinfo={
        Title={\thetitle},
        Author={\theauthor}
    }
}

% Visual configuration
\OnehalfSpacing
\setlength{\parindent}{1cm}
\setlength{\parskip}{\baselineskip}

\begin{document}
\imprimircapa

\tableofcontents

\chapter{Introdução}
Descrição do trabalho e sua contextualização com a Arquitetura de Computadores
Neste projeto, foi implementação de um Simulador de Associatividade Adaptativa que, dado uma lista de endereços no formato hexadecimal, tenta acessá-los na memória cache por 3 métodos diferentes (vistos em sala de aula): \textbf{Mapeamento Direto}, \textbf{Mapeamento Totalmente Associativo} e \textbf{Mapeamento Parcialmente Associativo}.

\begin{description}
\item[Mapeamento Direto]
A técnica de mapeamento direto consiste em acessar um endereço na memória utilizando congruência modular, isto é, caso a memória tenha $ n $ posições e o endereço seja $ x $, ele deve ser mapeado na posição $ x \bmod n $. Possuindo, assim, uma complexidade constante ($ \Theta (1) $).
\item[Mapeamento Totalmente Associativo]
Essa é a técnica mais ``trivial'', pois consiste em percorrer linearmente a cache à procura do endereço. Dessa forma, sua complexidade é dada por $ \mathcal{O} (n) $.
\item[Mapeamento Parcialmente Associativo]
Nesse método, há uma ``mistura'' entre o Mapeamento Totalmente Associativo e o Direto, pois a cache é particionada, formando assim um conjunto de ``pequenas caches'' de tamanhos menores. Assim, para fazer a busca de um elemento, é feito o cálculo de congruência modular entre o número de partições e então o elemento é procurado linearmente nesta seção menor, fazendo com o que a complexidade seja $ \mathcal{O} (\lceil n/k \rceil) $ sendo $ n $ o tamanho da cache e $ k $ o número de seções que a cache foi dividida.
\end{description}

Neste simulador, quando a ``taxa de \textit{miss}'' (falhas na busca / número de buscas) é superior a um percentual $ x $ (informado pelo usuário), o tipo de mapeamento deve ser alterado. E a cada busca deve ser dada a opção de exibição do estado atual da memória cache.

\chapter{Solução Implementada}
O algoritmo foi implementado utilizando basicamente 2 classes, uma para a memória cache e outra para o simulador em si.
A classe \code{Cache} consiste em um \code{std::vector} e uma variável para armazenar o seu tamanho. Seus construtor recebe apenas o seu tamanho como argumento e seus outros métodos são \code{getSize()}, \code{getValue()}, \code{setValue()} e \code{show()} que servem, respectivamente, para retornar o tamanho da cache, o valor armazenado em uma determinada posição, alterar o valor de uma posição e mostrar uma representação visual de seu conteúdo no console.

A outra classe utilizada foi a \code{MappingSimulator} que contém toda a lógica do simulador. Ela possui uma \code{Cache} e algumas variáveis para armazenar os dados de taxa máxima de miss, contador de miss, contador de acessos e o tipo atual de mapeamento. O seu construtor é chamado com 2 argumentos, o tamanho da Memória Cache e a taxa máxima de miss suportada para que o programa altere o tipo de mapeamento. Seus métodos públicos são \code{access()} e \code{show()}. O primeiro é o método principal do simulador, em que o usuário passa como parâmetro um endereço de memória em hexadecimal (armazenado como uma \code{std::string}) e ele se encarrega de verificar o método de mapeamento a ser utilizado e fazê-lo da forma correta. Já o método \code{show()} é apenas um alias para o método \code{show()} da classe \code{Cache}, que não pode ser acessado pelo cliente pois a \code{Cache} é um membro privado da classe.
Além desses dois métodos, essa classe conta também com o método privado \code{search\_val()} que recebe como parâmetro o tipo de mapeamento a ser utilizado na busca, o valor a ser buscado e uma variável para armazenar a melhor posição para inserir o elemento caso ele não tenha sido encontrado (\textit{miss}).

Para obter a posição de inserção do elemento quando houver um \textit{miss} (quando o mapeamento for associativo), foi utilizado um algoritmo que se assemelha ao FIFO (\textit{First in - First out}), no entanto foi desenvolvido com base na aritmética modular.
Os casos poderiam ser divididos como:
\begin{itemize}
\item Houve um \textit{miss} mas ainda existem espaços livres na área de inserção: O algoritmo retorna a posição correspondente à primeira área livre.
\item Houve um \textit{miss} e não existem mais espaços disponíveis na área de inserção: Sendo $ ini $ = posição de início da área de inserção , $ area $ = tamanho da área de inserção e $ cont $ = contador de acessos à memória , o algoritmo retorna a posição $ ini + (cont \bmod area) $. Pois assim, se a área for de tamanho 4, e iniciar no 0, e já ocorreram 4 inserções, o próximo elemento a ser substituído da cache é o primeiro. (Com essa solução, pode não haver uma precisão muito alta, no entanto há uma economia na memória utilizada).
\end{itemize}

\section{Organização do Código}
O projeto foi estruturado em 5 pastas, são elas:
\begin{itemize}
\item \textbf{bin}:           Onde os arquivos executáveis/binários são salvos.
\item \textbf{build}:         Onde os arquivos objeto são salvos.
\item \textbf{data}:          Armazena os arquivos de entrada (conjuntos de endereços).
\item \textbf{documentation}: Contém todos os arquivos da documentação gerada pelo Doxygen.
\item \textbf{includes}:      Estão localizados todos os cabeçalhos (\textit{headers}) das classes e funções.
\item \textbf{src}:           Código fonte do projeto e a implementação dos métodos das classes e das funções.
\end{itemize}

\section{Execução do Código}
Para compilar esse programa, tudo o que você precisa fazer é digitar:

\code{\textdollar make clean}

\code{\textdollar make}

E para executá-lo:

\code{\textdollar ./bin/memory\_mapping arquivo\_enderecos taxa\_de\_miss [tamanho\_da\_cache]}

Onde o \code{arquivo\_endereços} representa um arquivo com todos os endereços a serem buscados pelo programa (um por linha), a \code{taxa\_de\_miss} é o percentual (em ponto flutuate) que definirá a troca do tipo de mapeamento e o \code{tamanho\_da\_cache} é um parâmetro opcional para definir o tamanho da memória cache (no mínimo 4 posições), caso não seja informado, ele assumirá um valor padrão de 16 posições.


\chapter{Análise de Resultados}
Para fazer a análise, o algoritmo foi modificado, de forma a evitar que houvesse a troca de tipo de mapeamento.
Foi utilizado como entrada um arquivo com 184 endereços de memória ordenados randomicamente e uma memória cache de 64 posições (para que a comparação pudesse ser realizada, foi utilizada a mesma entrada em todas as execuções).

Dessa forma, foram obtidos os seguintes resultados relacionados à ``taxa de \textit{miss}'' de cada algoritmo:

\begin{description}
\item[Mapeamento Direto]
A ``taxa de \textit{miss}'' final para o arquivo de entrada foi de aproximadamente $ 23.913\% $
\item[Mapeamento Parcialmente Associtivo]
Para esse tipo de mapeamento, foi obtida uma ``taxa de \textit{miss}'' na ordem de $ 10.7486\% $
\item[Mapeamento Totalmente Associativo]
Utilizando essa técnica, a ``taxa de \textit{miss}'' obtida foi de $ 7.6087\% $
\end{description}

O algoritmo foi, novamente, modificado de forma que nenhuma informação fosse escrita no terminal (eliminando assim o tempo decorrido para escrita de informações na tela) e foi adicionado um laço para que o algoritmo fosse executado 100,000 vezes seguidas e gerando o tempo de execução médio (utilizando o cálculo do Desvio Padrão).

Dessa forma, ao fazer a análise dos algoritmos por tempo de execução, os resultados obtidos foram:

\begin{description}
\item[Mapeamento Direto]
Utilizando apenas esse tipo de mapeamente, obtivemos um tempo médio de execução de 58.6573 microssegundos.
\item[Mapeamento Parcialmente Associativo]
Para esse tipo de mapeamento, o tempo médio de execução foi de 84.2167 microssegundos.
\item[Mapeamento Totalmente Associativo]
Nesse tipo de mapeamento, o tempo médio de execução obtido foi de 174.063 microssegundos.
\end{description}

Analisando os resultados obtidos, podemos perceber que a maior ``taxa de \textit{miss}'' foi encontrada no Mapeamento Direto (havendo um ``empate'' entre os mapeamentos associativos) que pode ser justificado por cada elemento buscado ter apenas uma posição possível para ser acessada na memória (um número congruente a ele que esteja entre 0 e o tamanho de memória).

No entanto, quando comparamos o tempo de execução, o Mapeamento Direto se sobressaiu, possuindo um tempo médio de execução na casa dos 60 microssegundos, justificado por sua complexidade ser $ \Theta (1) $ (constante). Já o mapeamento Parcialmente Associativo obteve um tempo médio próximo ao obtido pelo Mapeamento Direto, com seus 84 microssegundos, devido a sua complexidade de acesso ser $ \mathcal{O} (\lceil n/k \rceil) $.
Na última posição ficou o Mapeamento Totalmente Associativo, resultado esse já esperado tendo em vista a sua complexidade $ \mathcal{O} (n) $, com o tempo na faixa de 174 microssegundos.


\chapter{Conclusão}
Após a uma análise dos resultados obtidos, pode-se concluir que, todos os tipos de mapeamento (direto, totalmente associativo e parcialmente associativo) possuem suas vantagens e desvantagens.

No caso do Mapeamento Direto, a vantagem é o tempo de acesso (e complexidade), no entanto, como a sua busca possui complexidade constante (pois olha em apenas um endereço) acaba por ter uma maior taxa de miss, possuindo assim um menor aproveitamento da memória.

O Mapeamento Totalmente Associativo é exatamente o oposto do Mapeamento Direto, pois consegue aproveitar a memória ao máximo e, por conta disso, o tempo de execução fica debilitado.

De acordo com as execuções feitas, o que possui um melhor equilíbrio entre o tempo de execução e o aproveitamento de memória é o Mapeamento Parcialmente Associativo, pois mescla características do Mapeamento Direto (congruência modular para diminuir o espaço de busca) e do Totalmente Asssociativo (busca linear em várias posições de memória).

Portanto, pode-se concluir que, em termos gerais, o Mapeamento que possui o melhor equilíbrio entre tempo de acesso e \textit{miss rate} é o \textbf{Mapeamento Parcialmente Associativo}, no entanto, ``cada caso é um caso'' e, dependendo do projeto, pode ser que um outro tipo de mapeamento se enquadre melhor, necessitando de uma análise individual.


\begin{anexos}
\chapter{Exemplo de Arquivo de Entrada}

\begin{raw}
\noindent
\texttt{0x0E}\\
\texttt{0xED}\\
\texttt{0x39}\\
\texttt{0x39}\\
\texttt{0xAF}\\
\texttt{0x34}\\
\texttt{0x3F}\\
\texttt{0x39}\\
\texttt{0xED}\\
\texttt{0x3F}\\
\texttt{0x45}\\
\texttt{0x13}\\
\texttt{0x3F}\\
\texttt{0xAF}
\end{raw}

\end{anexos}

\end{document}
